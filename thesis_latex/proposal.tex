\documentclass[12pt]{article}

\usepackage{graphics}
\usepackage{epsfig}
\usepackage{times}
\usepackage{amsmath}
%\usepackage{float}  % for floating figures
\usepackage{floatrow} %for caption of figures on its side
\usepackage{natbib}  %for \citep
\usepackage{gensymb} %for \degree
\usepackage{subcaption} %for figures side by side

%%------------------------------------------------
%   For trackchanges package
%   Change the editor name as you like
%%------------------------------------------------
\usepackage[inline]{trackchanges}
\addeditor{XT}
\addeditor{EC}
%%------------------------------------------------
% Try one of these commands:
%     \add[editor]{added text}
%     \remove[editor]{removed text}
%     \change[editor]{removed text}{added text}
%     \note[editor]{note text}
%     \annote[editor]{text to annotate}{note text}
%%------------------------------------------------

% Borrowed the template from the CS dept. of Columbua University.
% <http://psl.cs.columbia.edu/phdczar/proposal.html>:
%
% The standard departmental thesis proposal format is the following:
%        30 pages
%        12 point type
%        1 inch margins all around = 6.5   inch column
%        (Total:  30 * 6.5   = 195 page-inches)
%
% For letter-size paper: 8.5 in x 11 in
% Latex Origin is 1''/1'', so measurements are relative to this.

\topmargin      0.0in
\headheight     0.0in
\headsep        0.0in
\oddsidemargin  0.0in
\evensidemargin 0.0in
\textheight     9.0in
\textwidth      6.5in

\title{{\bf 3D Numerical Models for Along-axis Variations in Diking} \\
\it Thesis proposal}

\author{ {\bf Xiaochuan Tian}  \\
Center for Earthquake Research and Information \\
The University of Memphis\\
{\small xtian@memphis.edu}
}
\date{\today}

\begin{document}
\pagestyle{plain}
\pagenumbering{roman}
\maketitle

\pagebreak
\begin{abstract}

Bathymetry observations show great variety of morphologies at Mid-ocean Ridges (MORs). Previous studies showed that the morphologies at slow spreading MORs are mainly controlled by the ratio between rates of magma supply and plate extension, denoted as ``M". The 2D framework is successful in explaining the various topography. However, since the magma supply varies along the ridge, the interactions between the tectonics and magmatism at MORs are inevitably 3D processes. Here, we propose to model the 3D processes with a 3D numerical modeling code, SNAC.

\end{abstract}

\pagebreak
\tableofcontents
\pagebreak

\cleardoublepage
\pagenumbering{arabic}

\section{Introduction}
\label{ch:intro}

Around 70\% of the Earth's crust is oceanic crust and the mid-ocean ridges (MORs) are the most dynamic places where new crust are forming with a multitude of earthquakes, volcanic processes constantly happening. To study how new crust is created and how MORs evolve is significant for Earth Sciences.  Geodynamics modeling along with a variety of geological, geophysical observation and lab experiment constraints can be used to study how the MORs system works under geological time scale of Million years which is impossible to be observed in the length of human history.

The MORs are the longest mountain chains on the Earth with a total length of about 60,000 km. Both along and across MORs axis, varied topography was observed from multi-beam bathymetric data. Three specific questions stimulate people's interests. First, what cause the distinct difference in axial topography between slow and fast spreading ridges. Second, for slow spreading ridges, why does topography  along ridge varies and how to explain many features observed. Third, why do Oceanic Core Complexes (OCCs) form and what is the mechanism. 

%\subsection{Background}
%\label{ch:back}
According to \citep{Fowler2004}, variations in  mid-ocean ridge morphologies are mainly controlled by four factors: magma supply, tectonic strain, hydrothermal circulation and spreading rate. \add[XT]{Among them, the spreading rate is the first order control factor.} Slow-to-intermediate spreading centers (half spreading rate less than 4cm/year) produce median valleys that are typically 10$\sim$20km wide and 1$\sim$2km deep (e.g., Mid-Atlantic Ridges, Figure 1(a)).Fast-spreading centers (half spreading rate greater than 5cm/year have axial highs that are 10$\sim$20 km wide, 0.3$\sim$0.5 km high (e.g., East Pacific Rise, Figure 1(b)).

\begin{figure}[H]
\centering
\begin{subfigure}{.5\textwidth}
  \centering
  \includegraphics[width=.8\linewidth]{fig1_1.png}
  \caption{\small{Slow spreading Mid-Atlantic Ridge}}
  \label{fig1_1}
\end{subfigure}%
\begin{subfigure}{.5\textwidth}
  \centering
  \includegraphics[width=.8\linewidth]{fig1_3.png}
  \caption{\small{Fast spreading East Pacific Rise}}
  \label{fig1_3}
\end{subfigure}
\caption{\small{Profiles of bathymetry across MORs.}}
\label{fig1}
\end{figure}
Slow spreading ridges exhibit along-axis variations in the width and depth of median valleys and the off-axis morphology.  Figure~\ref{fig2_1} shows that the topographic profile nearer to the center of the ridge segment (A-A') is rather symmetric and has higher frequency. The maximum relief is about 1km. In constrast, the near-tip profile (B-B') is asymmetric and has much lower frequency and a greater relief ($\sim$3km). The bathymetry and crustal thickness along the ridge valley also varies. From \citep{Chen1999}, the maximum along-axis variation in crustal thickness $\Delta H_{c}$ is linearly increasing with segment length $L$, and the relationship is $\Delta H_{c}(L)=0.0206L$ (Figure~\ref{fig3_1}).

\begin{figure}[H]
 \centering
  \includegraphics[scale=0.4]{fig2_1.png}
 \caption{\small{Two bathymetry cross-sections of Mid-Atlantic Ridge (MAR) with 10 times vertical exaggeration. A-A' is closer to the ridge segment center while B-B' is at the tip of the segment near the Atlantis Transform fault.}}
 \label{fig2_1}
\end{figure}

\begin{figure}[H]
 \centering
  \includegraphics[scale=0.3]{fig3_1.png}
 \caption{\small{Relationship between the maximum crustal thickness variations along a ridge segment and the segment length.The dashed line is the best-fit linear regression of the combined data. \citep{Chen1999}}}
 \label{fig3_1}
\end{figure}

%Moreover, a 3D view of B-B' area of Figure~\ref{fig2_1} shows the 20km wide, 25km long and 3km relief Atlantis Massif. The huge geologic structure is a window for learning lower crust and upper mantle because it is believed to be a result of exhumation of deeper material to the seafloor through tectonic processes.
%
%\begin{figure}[H]
% \centering
%  \includegraphics[scale=0.4]{fig4_1.png}
% \caption{\small{Zoom in B-B' area, a 3D view of Atlantis Massif.}}
% \label{fig4_1}
%\end{figure}

%\subsection{Related work}
%\label{ch:related}
Magma supply at mid-ocean ridges is mostly a passive process when no hot plume presents \citep{Fowler2004}. Hot mantle rises up \remove[XT]{only} to fill the vacated room being created by plate separation and decompression will lead to partial melting of the hot mantle. The melt upwells \remove[XT]{to the surface} due to both pressure difference and buoyancy from lateral density difference. \change[XT]{When the melt reaches the surface}{During the process, heat is conducted away to its colder surrounding rocks}, it then solidifies, forming the new crust at the spreading centers and releases the tension from far field stretching. 

The passive nature of magma supply results in the major difference between fast and slow spreading ridges that magma supply is always sufficient to accommodate all the tension built up for fast spreading ridges while is deficit for slow spreading ridges. \citep{Buck2005} explained the distinct difference between fast and slow spreading ridge topography through a 2D modeling study by showing the effect of the variable amount of plate motions accommodated by dikes on faulting patterns and ocean floor morphology. As shown in Figure~\ref{fig5_1}, for the fast spreading ridge, the axial high is generated by buoyancy from lateral density difference across ridge axis and for the slow spreading ridge, the median valley is formed due to near-axis normal faulting. In this study, a simple method for parameterizing \change[XT]{repeated diking}{the ratio between the rate of diking and plate spreading} has been established. \add[XT]{The ratio $M=\Delta\varepsilon_{xx}/2V_{x}$, where $\Delta\varepsilon_{xx}$ is the expanding strain due to the widening of diking in a unit time and $V_{x}$ is the half spreading rate of the MOR. Considering two end members: when $M=1$, diking will release all the tension results from plate separation; when $M=0$, no magma supply, tension from far field extension can only be accommodated by tectonics processes.} \annote[EC]{"M"}{Expand the description of the M-factor method. Give a formal definition and a mathematical expression here once and for all and stop putting quotation marks around M.}
\begin{figure}[H]
 \centering
  \includegraphics[scale=0.7]{fig5_1.png}
 \caption{\small{Upper one: modeling result for fast spreading agrees well with the observation of East Pacific Rise. Lower one: modeling result for slow spreading ridges agrees well with the bathymetry of Mid Atlantic Ridge. \citep{Buck2005}}}
 \label{fig5_1}
\end{figure}
\citet{Tucholke2008} expanded investigation on the role of M in the mid-ocean ridge mechanics. They focus on faulting behaviors of slow spreading ridges and find that the Oceanic Core Complexes (OCCs) are most likely to form when M varies from 0.3 to 0.5. As shown in Figure 5, when M=0.7, repeated diking pushes faults forming at the spreading center away from axis. \annote[XT]{As the off-axis faults are locked, a new near-axis fault forms.}{off-axis faults should lock after the new near-axis fault forms (how about change to:)``When the energy for maintaining the fault increases as it move off axis and exceeds the energy for breaking a new near-axis one, it will be replaced by the new fault."} When M=0.3$\sim$0.5, the normal faults remains active for a long time to become detachment faults, exhuming the lower crust and mantle materials to the seafloor. When M is less than 0.3, most of the tension is accommodated by tectonic process rather than by diking and as a result, faulting pattern is more complicated and unsteady.

The M-factor framework of the 2D models has been successful in explaining various observation of seafloor bathymetry across ridge axis. However, 2D models have limitations in studying the along ridge-axis interactions, especially when important variables are not constant along the ridge axis. The passive nature for magma supply results in a constantly sufficient and negligible variation in magma supply along fast spreading ridge axis. The topography along fast spreading ridges does not vary much. However, along the slow spreading ridges axis, controlling variables varies. Bathymetry, gravity anomaly, reflection and refraction seismology consistently show distinguishable variation in crustal thickness \citep{Ryan2009}; \citep{Chen1999}; \citep{Lin1990}; \citep{Tolstoy1993}. Assuming oceanic crust is mainly formed by upwelled magma at the ridge, variation in the thickness of the crust implies variation in magma supply. Thus, for slow-to-intermediate spreading ridges, the interactions between tectonics and magmatism at MORs are inevitably 3D processes and 3D numerical models are desirable for better understanding factors controlling both across- and along-ridge variations. 
\begin{figure}[H]
\floatbox[{\capbeside\thisfloatsetup{capbesideposition={right,bottom},capbesidewidth=5cm}}]{figure}[\FBwidth]
{\caption{\small A$\sim$F: Faulting behavior for different values of M. Geologic interpretation is superimposed on modeled distribution of strain rate. Dots show breakaways of initial faults. Dashed seafloor is original model seafloor, red dotted seafloor is formed dominantly by magmatic accretion, and solid bold is fault surface. Note that detachment faults in B and C are not interrupted by secondary faults. \citep{Tucholke2008}}}
 {\includegraphics[width=10cm]{fig6_1.png}} 
 \label{fig6_1}
\end{figure}


\break
\section{Proposed Work}
\label{ch:purpose}

At slow spreading ridges, other controlling variables such as hydrothermal cooling, thermal structures and even local spreading rate \citep{Baines2008} also varies both along and across the ridge axis and they are interrelated. But, we propose that the interaction between magmatism and tectonics which is quantified as the ratio M is the first order control over the topography evolution of MORs. 

The purpose of this thesis is to study how the along ridge axis varying M will make a contribution to the observed various topography. We will implement the 2D framework M into a 3D numerical modeling code SNAC \citep{Choi2008} and focus on the last two questions mentioned in the introduction. Hopefully, through systematically explaining the behaviors of the 3D models and comparing that with observations, we can understand better how the MORs magmatism and tectonics system works and build up a foundation for future 3D numerical modeling studies on this research topic.



\break
\subsection{Method of Approach}
\label{ch:method}

The idea is that we utilize geophysical observations and published results to constrain the setups of the 3D model. By iteratively minimizing the gap between results from models and observations, hopefully, a model with consistent results will be generated. Then, we will interpret the observations based on that model.

The numerical modeling code, SNAC, is an explicit Lagrangian finite element code. It is a reusable set of libraries or classes for a software system. It uses the energy-based finite difference method to solve the force balance equation for elasto-visco-plastic materials. Figure~\ref{fig7_1} shows major parts of the methods. For each time step, due to boundary conditions (Figure~\ref{fig8_1})(e.g., half spreading velocity $V_{x}$ prescribed on both side walls of the model, Winkler foundation at the bottom, simulated sea water hydrostatic pressure on the surface, free slip on all six walls), the stress state $\sigma_{ij}$ will be calculated from Momentum Balance Equation. Then, forces calculated from $\sigma_{ij}$ will be used to calculate velocity. Through this iterative process, the evolution of sea floor tectonics can be modeled. The 3D model is discretized into two sets of hexahedral elements so that faulting will not favor a specific direction. Rheology for the brittle crust will be modeled as elastic and plastic material controlled by constitutive equations and Mohr-Coulomb failure criterion while for ductile mantle, results of ductile behaviors of dry diabase examined by lab experiments will be applied \citep{Kirby1987}. For 3D diking processs, the expanding strain $\Delta\varepsilon_{xx}$ results from M will lead to extra-stresses in all three directions $\Delta\sigma_{xx}$, $\Delta\sigma_{yy}$ and $\Delta\sigma_{zz}$ based on the constitutive equations $\sigma_{ij}=\delta_{ij}\lambda\varepsilon_{ij}+2\mu\varepsilon_{ij}$.

\begin{figure}[H]
 \centering
  \includegraphics[scale=0.46]{fig7_1.png}
 \caption{\small{Essential theories for the numerical methods}}
 \label{fig7_1}
\end{figure}

\subsection{Model Setup}

I propose the model setup shown in Figure~\ref{fig8_1}. For temperature structure, depth from 0$\sim$6km, linear increases from 0 \degree C to 240 \degree C (remain low for rough approximation of hydrothermal circulation); depth from 6$\sim$20km, instantaneous cooling of a Semi-infinite Half-space (erf function) \citep{Turcotte2002}. For rheology, we will follow dry diabase power law rheology from lab experiments \citep{Kirby1987}. For boundary conditions, free slip on all boundaries, surface temperature is 0\degree C, bottom temperature is 1300 \degree C.

\begin{figure}[H]
 \centering
  \includegraphics[scale=0.5]{fig8_1.png}
 \caption{\small Preliminary model setup}
 \label{fig8_1}
\end{figure}


Although how the M varies along the ridge is not clear due to the complexity of the system and the difficulties of observation: timewise, extremely long geological timescale and spacialwise, deep under the sea water burying under the ridge. However, we do have constraints from a large sum of gravity and seismic surveys as well as geological drilling. Generally, at slow spreading ridges, magma supplies mostly at the center of the ridge segment and decreases towards the end of the segment \citep{Tolstoy1993,Chen1999}. There is also evidence for shorter wavelength of 10 to 20 km discrete focus of magma accretion along the ridge axis \citep{Lin1990}. Based on these constraints, we will test different scenarios of varying M along the ridge axis.

\note[EC]{Describe what parameters and what range of their values you are going to explore and why.} \note[XT]{this part is crucial for the project and will be finished after two work being done: 1) thoroughly parameters survey on models we have, based on that understanding will help we proposed what model parameters to be texted; 2) thorough literature review on how M should varies along ridge axis and maybe statistically study on topography along ridge axis from GeoMapApp. }



\break
\section{Work Plan}
\label{ch:plan}

Table \ref{tab:plan} shows what have been done so far and what still need to be finished for the completion of the research.

\begin{table}[hc]
\begin{small}
\begin{center}
\begin{tabular}{|l|p{7cm}|l|}
\hline
Timeline & Work & Progress\\
\hline
Dec., 2013$\sim$Jan., 2014& Understand the basics of MORs tectonics and magmatism system & completed\\ \hline
Jan., 2014$\sim$March, 2014& run 2D models in FLAC to understand the framework& completed\\ \hline
March, 2014$\sim$May, 2014& Learn C programing and continuum mechanics and understand the 3D code& completed\\ \hline
May, 2014$\sim$July, 2014& Benchmark pseudo 3D code with previous modeling results& completed\\ \hline
July, 2014$\sim$Sept., 2014& Implement varying M into 3D code SNAC& completed\\ \hline
Sept., 2014$\sim$Nov., 2014& Run preliminary 3D models and explain the model behaviors& completed\\ \hline
Nov., 2014$\sim$Dec., 2014& Prepare AGU conference poster and presentation& completed\\ \hline

Jan., 2015$\sim$Feb., 2015  & Run different scenarios of M and systematically examine how M affect the topography & ongoing\\ \hline
Feb. 2015$\sim$March, 2015 & Thesis writting & ongoing\\ \hline
April. 2015 & Thesis defense & \\ \hline
\end{tabular}
\end{center}
\end{small}
\caption{An overview of what have been done and plan for what need to be done}
\label{tab:plan}
\end{table}

%\pagebreak
%\section{Budget}
%\label{ch:budget}
%Since observations(topography data of mid-ocean ridges) are handy from GeoMapApp, we don’t need to conduct side scan sonar survey in the deep ocean. GeoMapApp can provide very good bathymetry data and images for this research topic. The software as well as the data are free. We might need some other geophysical evidence (seismology, gravity, resistivity, magnetic etc.) to further support results of the model. They can be attained from other people’s research results. Most of the time will be invested in an iteration process—computer calculations on the 3D model and adjustment on the settings of the model and then calculate again based on previous results. Jobs are run on Super computers Stampede and Penguin. They are our HPC expenses.


\break
\begin{footnotesize}
\bibliographystyle{abbrvnat}
\bibliography{Thesis_Proposal.bib}


\end{footnotesize}

\end{document}