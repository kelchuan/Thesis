\begin{center}
\textbf{\textit{Abstract}}
\end{center}

\vspace{0.5cm}

\begin{singlespace*}
Tian, Xiaochuan. M.S. The University of Memphis. May 2015 Master of Science. 3D Numerical Models for Along-axis Variations in Diking at Mid-Ocean Ridges. Major Professor: Dr. Eunseo Choi.
\end{singlespace*}

\vspace{0.5cm}

Bathymetry reveals diverse morphologies at Mid-ocean Ridges (MORs). Previous studies show that the morphologies at slow spreading MORs are mainly controlled by the ratio (M) between rates of magma supply and plate extension. 2D models successfully explain many cross-sectional observations across the ridge axis. However, magma supply varies along the ridge and the interaction processes between the tectonics and magmatism at MORs are inevitably three dimensional. We investigate the consequences of this along-axis variation in diking in terms of faulting patterns and the associated structures with a 3D parallel geodynamic modeling code, SNAC. Many structural features observed are produced. We also proposed asynchronous faulting induced tensile failure as a new possibility for explaining corrugations. $\bar{M} =$0.6425 is suggested as a boundary value for separating abyssal hills and oceanic core complexes (OCCs) formation. Previous inconsistency for OCCs formation between 2D model results (M = 0.3$\sim$0.5) and field observations that M can extend beyond the range is reconciled by our 3D along ridge coupling argument.  

