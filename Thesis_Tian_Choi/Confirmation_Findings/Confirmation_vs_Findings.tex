% Created 2015-07-16 Thu 05:53
\documentclass[11pt]{article}
\usepackage[utf8]{inputenc}
\usepackage[T1]{fontenc}
\usepackage{fixltx2e}
\usepackage{graphicx}
\usepackage{longtable}
\usepackage{float}
\usepackage{wrapfig}
\usepackage{rotating}
\usepackage[normalem]{ulem}
\usepackage{amsmath}
\usepackage{textcomp}
\usepackage{marvosym}
\usepackage{wasysym}
\usepackage{amssymb}
\usepackage{hyperref}
\tolerance=1000
\author{Xiaochuan Tian}
\date{\today}
\title{Confirmation\_vs\_Findings}
\hypersetup{
  pdfkeywords={},
  pdfsubject={},
  pdfcreator={Emacs 24.5.1 (Org mode 8.2.10)}}
\begin{document}

\maketitle
\tableofcontents

\section{Findings}
\label{sec-1}
1.1 and 1.2 need field evidences. 
\subsection{Along ridge coupling for reconciling discrepancy between 2D model M(0.3\textasciitilde{}0.5) and field observation.}
\label{sec-1-1}
\subsection{Asynchronous faulting induced isochron-parallel tensile failure as a mechanism for corrugations.}
\label{sec-1-2}

\subsection{From major features}
\label{sec-1-3}
Comparing model results and field observation can help to infer historical tectonics and magmatism evolution.
\subsubsection{mass wasting}
\label{sec-1-3-1}
\subsubsection{hourglass median valley}
\label{sec-1-3-2}
\section{Confirmation}
\label{sec-2}
\subsection{Average M = 0.6425 for separating abyssal hills and OCC formation.}
\label{sec-2-1}
This is first mentioned by [Buck et al., 2005] in 2D version, we confirm that when M increase, faulting begin to alternate. 
We update on: first, it is 3D version average M for varying M along the ridge; Second, it is very sensitive to weakening rate because only type 2 (slower) weakening results in fault alternation. Further investigation needed to be done on different functional forms, ranges of M variations and different weakening rates. 
\subsection{From major features}
\label{sec-2-2}
\subsubsection{Inward fault jump}
\label{sec-2-2-1}
First mentioned by [Tucholke et al., 1998], but first time 3D modeling. It provide an explanation for brother domes.
\subsubsection{Mullion structure}
\label{sec-2-2-2}
Due to Anastomosing (Smith et al., 2014) or continuous casting model (Spencer, 1999). Still first time in 3D modeling.
% Emacs 24.5.1 (Org mode 8.2.10)
\end{document}
