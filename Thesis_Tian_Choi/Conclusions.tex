\pagebreak
\section{Conclusions}
I model in 3D how M varying along a mid-ocean ridge segment can control the interactions between tectonics and magmatism and thereby produce the major bathymetric features observed on the seafloor.
Six commonly observed MOR features are produced in our model which are inward fault jump, fault alternation, mass wasting, hourglass median valley, corrugation, and mullion structure. I show that, by comparing these features of the models and field observations, faulting history and spatial and temporal variation of magma supply can be inferred for a ridge segment.

Three controlling parameters are investigated. They are three ranges of M (i.e., M28, M57 and M58); three types of functional forms of M variation (i.e., linear, sinusoidal and sqaure root) as well as two types of weakening rates (i.e., faster type 1 and slower type 2). Although different structural features are generated by different functional forms and ranges of M variations along the ridge, the average M value ($\bar{M}$) along the whole ridge segment controls off-axis morphologies. When $\bar{M} >$ 0.6425 with type 2 weakening, spreading occurs symmetrically and short-wavelength abyssal hills are produced. In constrast, when $\bar{M} \le$ 0.6425, a long lasting detachment fault forms, which can exhume ultramafic rocks to the seafloor producing a domal OCC. Also, our 3D model results resolve the discrepancy between previous 2D model studies and field observations that model studies suggest OCC is produced when M varies from 0.3 to 0.5 but field observation reveals cases that OCC forms with observed M beyond both lower and upper limits. We propose a new perspective that along ridge coupling can explain why OCCs can form for M smaller than 0.3 or greater than 0.5. Finally, we propose that the asynchronous faulting can generate along-ridge tensile stresses of a large enough magnitude to cause tensile failure on the surface of the dome of an OCC, producing corrugations widely observed on the OCCs in nature.




