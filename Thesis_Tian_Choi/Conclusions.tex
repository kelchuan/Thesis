\pagebreak
\section{Conclusions}
To the first time, we model in 3 dimensions on how varying M value (magma supply) along the the mid-ocean ridge segment can control the interaction system between tectonics and magmatism and how is it responsible for the major bathymetric features observed on the seafloor.

Six commonly observed MOR features (i.e. inward fault jump, fault alternation, mass wasting, hourglass median valley, corrugation and mullion structre) are produced by our model. By comparing the model results and local field observations, faulting evolution history as well as spatial and temporal variation of magma supply can be inferred.

Three controlling parameters are investigated. They are three ranges of M (i.e. M28, M57 and M58); three type of functional forms of M variation (i.e. linear, sinusoidal and sqaure root) as well as two types of weakening rates (i.e. faster type 1 and slower type 2). As result shows, besides many different structural features generated by different functional forms and ranges of M variations along the ridge, the average M value ($\bar{M}$) along the whole ridge segment is one of the major value that is responsible for the two end members of off-axis morphologies, i.e. when $\bar{M} >$ 0.6425 with type 2 weakening, the model generates symmetrial spreading, high frequency abyssal hills while when $\bar{M} <=$ 0.6425, fault tends to rotate to a low angle, long lasting detachment fault that can exhume ultramafic rocks to the seafloor producing a domal OCC. Also, our 3D model results resolve the discrepancy between previous 2D model studies and field observations that model studies suggest OCC is produced under M values between 0.3 to 0.5 while field observation reveals cases of OCC forming under observed M beyond both lower and upper limits. Base on this thesis study, we propose a new perspective for explaining cases when OCC forms with local M value is lower than 0.3 or larger than 0.5 by the along ridge coupling argument.

In addition, we propose a new hypothesis of the asynchronous faulting induced tensile failure as one possible formation mechanism for the enigmatic corrugations widely observed on the surface of OCCs.




