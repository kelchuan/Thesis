\pagebreak
\section{Conclusions}
\remove[EC]{For the first time, we}{I} model in 3D how \remove[EC]{varying} M \remove[EC]{value (magma supply)} \add[EC]{varying} along \change[EC]{the}{a} \remove[EC]{the} mid-ocean ridge segment can control the interaction\add[EC]{s} between tectonics and magmatism and \add[EC]{thereby produce} the major bathymetric features observed on the seafloor.
Six commonly observed MOR features \annote[EC]{(}{STOP USING PARENTHESES LIKE THIS!!!!!!} \remove[EC]{i.e. inward fault jump, fault alternation, mass wasting, hourglass median valley, corrugation and mullion structure)} are produced \change[EC]{by}{in} our model \add[EC]{which are} inward fault jump, fault alternation, mass wasting, hourglass median valley, corrugation, and mullion structure. 
\add[EC]{I show that,} by comparing \add[EC]{these features of} the model\change[EC]{ results}{s} and \add[EC]{those of the} \annote[EC]{local}{What do you mean by ``local''?} field observations, faulting \remove[EC]{evolution} history \change[EC]{as well as}{and} spatial and temporal variation of magma supply can be inferred \add[EC]{for a ridge segment}.

Three controlling parameters are investigated. They are three ranges of M (i.e. M28, M57 and M58); three types of functional forms of M variation (i.e. linear, sinusoidal and sqaure root) as well as two types of weakening rates (i.e. faster type 1 and slower type 2). \change[EC]{As result shows, a}{A}lthough different structural features are generated by different functional forms and ranges of M variations along the ridge, the average M value ($\bar{M}$) along the whole ridge segment \change[EC]{is the major value that is responsible for the two end members of}{controls} off-axis morphologies\change[EC]{, i.e. w}{. W}hen $\bar{M} >$ 0.6425 with type 2 weakening, \change[EC]{the model generates symmetrial spreading, short wavelength abyssal hills}{spreading occurs symmetrically and short-wavelength abyssal hills are produced.} \change[EC]{whereas}{In constrast,} when $\bar{M} \le$ 0.6425, \remove[EC]{faults tend to rotate to a low angle,} \add[EC]{a} long lasting detachment fault \add[EC]{forms,} \change[EC]{that}{which} can exhume ultramafic rocks to the seafloor producing a domal OCC. Also, our 3D model results resolve the discrepancy between previous 2D model studies and field observations that model studies suggest OCC is produced when M varies from 0.3 to 0.5 but field observation reveals cases that OCC forms with observed M beyond both lower and upper limits. \change[EC]{Based on this thesis study, w}{W}e propose a new perspective \remove[EC]{for explaining cases when OCC forms with local M value is lower than 0.3 or higher than 0.5 by} the along ridge coupling \change[EC]{argument}{can explain why OCCs can form for M smaller than 0.3 or greater than 0.5}. \change[EC]{In addition}{Finally}, we propose \change[EC]{a new hypothesis of the asynchronous faulting induced tensile failure as one possible formation mechanism for the enigmatic corrugations widely observed on the surface of OCCs}{that the asynchronous faulting can generate along-ridge tensile stresses of a large enough magnitude to cause tensile failure on the surface of the dome of an OCC, producing corrugations widely observed on the OCCs in nature}.




